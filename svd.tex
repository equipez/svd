%\documentclass[11pt,a4paper,draft]{article}  % Use this line if this document is a draft
\documentclass[11pt,a4paper]{article}  % Use this line if this document will be released
\usepackage{ifdraft}

% Bibliography
\newcommand{\bibfile}{ref-svd.bib}  % Name of the BibTeX file.
\newcommand{\universalbib}{zref.bib}
\ifdraft{\IfFileExists{./\universalbib}{\renewcommand{\bibfile}{\universalbib}}{}}{}


% Geometry
%\voffset=-1.5cm \hoffset=-1.4cm \textwidth=16cm \textheight=22.0cm  % Luis' setting
\usepackage[a4paper, textwidth=16.0cm, textheight=22.0cm]{geometry}


% Basic packages
\usepackage{amsmath,amsthm,amssymb,amsfonts}
\usepackage{empheq}
\usepackage{lscape}
\usepackage{longtable}
\usepackage{color}
\usepackage[bbgreekl]{mathbbol}
\DeclareSymbolFontAlphabet{\mathbbm}{bbold}
\DeclareSymbolFontAlphabet{\mathbb}{AMSb}
\usepackage{bbm}
\usepackage{rotating}
\usepackage{multirow,booktabs}
\usepackage{upgreek}
\usepackage{accents}
\usepackage{xspace}
\usepackage[en-US]{datetime2}


% Graph, tikz and pgf
%\usepackage{subfigure}
\setlength{\unitlength}{1mm}  %The \unitlength command is a Length command. It defines the units used in the Picture Environment.
\usepackage{graphicx}
\usepackage{tikz,tikzscale,pgf,pgfarrows,pgfnodes,filecontents,tikz-cd,}
\usetikzlibrary{arrows,arrows.meta,patterns,positioning,decorations.markings,shapes}
\usepackage{pgfplots}
\usepackage{pgfplotstable}
\usepackage[justification=centering]{caption}
\usepgfplotslibrary{fillbetween}
\pgfplotsset{compat=1.11}


\usepackage[normalem]{ulem}
\usepackage[toc,page]{appendix}
\renewcommand{\appendixpagename}{\Large{Appendix}}
\renewcommand{\appendixname}{Appendix}
\renewcommand{\appendixtocname}{Appendix}
%\usepackage{sectsty}
\setcounter{tocdepth}{2}
\numberwithin{equation}{section}


% Turn off some unharmful warnings
\usepackage{silence}
\WarningFilter{latex}{Writing or overwriting file} % Mute the warning about 'writing/overwriting file'
\WarningFilter{latex}{Writing file} % Mute the warning about 'writing/overwriting file'
\WarningFilter{latex}{Tab has} % Mute the warning about 'Tab has been converted to Blank Space'


% Enumerate and itemize
\usepackage{eqlist}
\usepackage{enumitem}
\setlist[itemize]{leftmargin=*}
\setlist[enumerate]{leftmargin=*}


% Hyperref
\definecolor{darkblue}{rgb}{0,0.1,0.5}
\definecolor{darkgreen}{rgb}{0,0.5,0.1}
\usepackage{hyperref}
\hypersetup{ colorlinks,%
linkcolor=darkblue,%
anchorcolor=darkblue,
citecolor=darkblue,%
urlcolor=darkblue}
\newcommand{\red}[1]{\textcolor{red}{#1}}
\newcommand{\blue}[1]{\textcolor{blue}{#1}}
\ifdraft{\usepackage{refcheck}}{} % Check unused labels
\usepackage{url}


% Algorithm environment
\usepackage[section]{algorithm}
\usepackage{algpseudocode,algorithmicx}
\newcommand{\INPUT}{\textbf{Input}}
\newcommand{\FOR}{\textbf{For}~}
\algrenewcommand\algorithmicrequire{\textbf{Input:}}
\algrenewcommand\algorithmicensure{\textbf{Output:}}
\algrenewcommand\alglinenumber[1]{\normalsize #1.}
\newcommand*\Let[2]{\State #1 $=$ #2}


% Theorem-like environments
\newtheorem{theorem}{Theorem}%[section]
\newtheorem{acknowledgement}{Acknowledgement}%[section]
\newtheorem{alg}{Algorithm}%[section]
\newtheorem{axiom}{Axiom}%[section]
\newtheorem{case}{Case}%[section]
\newtheorem{claim}{Claim}%[section]
\newtheorem{conclusion}{Conclusion}%[section]
\newtheorem{condition}{Condition}%[section]
\newtheorem{conjecture}{Conjecture}%[section]
\newtheorem{corollary}{Corollary}%[section]
\newtheorem{criterion}{Criterion}%[section]
\newtheorem{exercise}{Exercise}%[section]
\newtheorem{lemma}{Lemma}%[section]
\newtheorem{notation}{Notation}%[section]
\newtheorem{problem}{Problem}%[section]
\newtheorem{proposition}{Proposition}%[section]
\newtheorem{remark}{Remark}%[section]
\newtheorem{solution}{Solution}%[section]
\newtheorem{assumption}{Assumption}%[section]
\newtheorem{summary}{Summary}%[section]
\newtheorem{note}{Note}%[section]
\newtheorem{doubt}{Doubt}%[section]
\newtheorem{properties}{Properties}%[section]
\newtheorem{example}{Example}%[section]
\newtheorem{stronconv}{Strong Convergence}%[section]
\theoremstyle{definition}
\newtheorem{definition}{Definition}%[section]
% proof
\usepackage{xpatch}
\xpatchcmd{\proof}{\itshape}{\normalfont\proofnamefont}{}{}
\newcommand{\proofnamefont}{\bfseries}


% Fine tuning
\usepackage{microtype}
\usepackage[nobottomtitles*]{titlesec} % No section title at the bottom of pages
% Prevent footnote from running to the next page
\interfootnotelinepenalty=10000
% No line break in inline math
\interdisplaylinepenalty=10000
\relpenalty=10000
\binoppenalty=10000
% No widow or orphan lines
\clubpenalty = 10000
\widowpenalty = 10000
\displaywidowpenalty = 10000


% Operators, commands
\usepackage{relsize}
\usepackage{nccmath}
\DeclareMathOperator*{\mcap}{\medmath{\bigcap}}
\DeclareMathOperator*{\mcup}{\medmath{\bigcup}}
%\renewcommand{\cap}{\mathsmaller{\bigcap}}
%\renewcommand{\cap}{\mcap}

\def\RR{\mathbb{R}}
\def\BB{\mathcal{B}}
\def\SS{\mathbb{S}}
\def\ZZ{\mathbb{Z}}
\def\NN{\mathbb{N}}
\def\FF{\mathbb{F}}
\def\CC{\mathbb{C}}
\newcommand{\sss}[1]{{\scriptscriptstyle{#1}}}
\newcommand{\sups}[1]{{#1}}
\newcommand{\sK}{{\scriptscriptstyle{K}}}
\newcommand{\sT}{{\scriptscriptstyle{T}}}
\DeclareMathOperator{\tr}{tr}
\newcommand{\fro}{{\scriptscriptstyle{\textnormal{F}}}}
\newcommand{\trs}{{\scriptscriptstyle{\mathsf{T}}}}
\newcommand{\hmt}{{\scriptscriptstyle{{\mathsf{H}}}}}
\newcommand{\pin}{{\scriptscriptstyle{{\mathsf{+}}}}}
\newcommand{\inv}{{-1}}
\newcommand{\adj}{*}

\DeclareMathOperator{\sort}{sort}
\DeclareMathOperator*{\Argmax}{Argmax}
\DeclareMathOperator*{\Argmin}{Argmin}
\DeclareMathOperator*{\argmax}{argmax}
\DeclareMathOperator*{\argmin}{argmin}
\DeclareMathOperator{\Span}{span}
\DeclareMathOperator{\med}{med}
\DeclareMathOperator{\essinf}{essinf}
\newcommand{\rank}{\mathrm{rank}}
\newcommand{\range}{\mathrm{range}}
\newcommand{\ind}[2]{\operatorname{\mathbbm{1}}\;\!\!\!\big(#2\in#1\big)}
\newcommand{\diag}{\operatorname*{diag}}
\newcommand{\Diag}{\operatorname*{Diag}}
\newcommand{\im}{\operatorname{im}}
\newcommand{\diam}{\operatorname{diam}}
\newcommand{\dist}{\operatorname{dist}}
\newcommand{\disth}{{\operatorname{\updelta_{\sss{H}}}}}
\newcommand{\Pred}{\mathrm{Pred}}
\newcommand{\cs}{\text{c}}
\newcommand{\hp}{\circ}
\newcommand{\card}{{\rm card}}
\newcommand{\fr}{\operatorname{fr}}
\newcommand{\sg}[1]{\bf { #1 }}
\newcommand{\ceil}[1]{ {\lceil{#1}\rceil} }
\newcommand{\floor}[1]{ {\lfloor{#1}\rfloor} }
%\renewcommand{\emph}{\textbf}
%\newcommand{\ones}{\mathbbm{1}}
\newcommand{\ones}{1}
\newcommand{\cc}{\sss{\textnormal{C}}}
\newcommand{\dec}{\sss{\textnormal{D}}}
\newcommand{\cauchy}{\sss{\textnormal{C}}}
\newcommand{\scauchy}{\sss{\textnormal{S}}}
\newcommand{\etc}{{etc.}}
\newcommand{\ie}{{i.e.}}
\newcommand{\eg}{{e.g.}}
\newcommand{\etal}{{et al.}}
%\newcommand*{\defeq}{\stackrel{\mbox{\normalfont\tiny{\textnormal{def}}}}{=}}
\newcommand\defeq{\mathrel{\overset{\makebox[0pt]{\mbox{\normalfont\tiny\sffamily def}}}{=}}}
\newcommand{\crit}{\textnormal{crit}}
\newcommand{\rsg}{\hat{\partial}}
\newcommand{\gsg}{\partial}
\newcommand{\dom}{\textnormal{dom}}
\newcommand{\tf}{{\textnormal{f}}}
\newcommand{\tg}{{\textnormal{g}}}
\newcommand{\ts}{{\textnormal{s}}}
\newcommand{\st}{\textnormal{s.t.}}
\newcommand{\me}{\mathrm{e}}
\newcommand{\md}{\mathrm{d}}
\newcommand{\mi}{\mathrm{i}}
\newcommand{\lev}{\mathrm{lev}}
\newcommand{\bA}{\mathbf{A}}
\newcommand{\bx}{\mathbf{u}}
%\newcommand{\bb}{\mathbf{f}}
\newcommand{\bb}{\mathbf{r}}
\newcommand{\nov}{n_{\textnormal{o}}}
\newcommand{\MATLAB}{\textsc{Matlab}\xspace}
\newcommand{\rpss}{{SS}}
\newcommand{\rdfs}{{DF}}
\newcommand{\pfs}{{FS}}
\newcommand{\cg}{{CG}}
\newcommand{\hem}{{HEM}}
\newcommand{\prblm}{\texttt}
\DeclareMathAlphabet{\mathsfit}{T1}{\sfdefault}{\mddefault}{\sldefault}
\SetMathAlphabet{\mathsfit}{bold}{T1}{\sfdefault}{\bfdefault}{\sldefault}
\newcommand{\prbb}{\mathsfit{p}}
\newcommand{\pp}{\mathsf{p}}
\newcommand{\qq}{\mathsf{q}}
\newcommand{\ttt}{\mathsfit{t}}
\newcommand{\tol}{\varepsilon}
\newcommand{\bt}{\mathbf{t}}
\newcommand{\br}{\mathbf{r}}
\newcommand{\dd}{\mathbf{d}}
\newcommand{\ii}{\mathbf{i}}
\newcommand{\jj}{\mathbf{j}}
\newcommand{\xx}{\mathbf{x}}
\renewcommand{\pp}{\mathbf{p}}
\renewcommand{\ggg}{\mathbf{g}}
\newcommand{\GG}{\mathbf{G}}
\renewcommand{\Pr}{\mathbb{P}}
\newcommand{\iid}{\text{i.i.d.}}
\newcommand{\integer}{\textrm{I}}
\newcommand{\octave}{\mbox{GNU Octave}\xspace}
\newcommand{\gradp}{\nabla_{\!\sss{P_k}}}
\newcommand{\lims}{\mathcal{L}}

% mathlcal font
\DeclareFontFamily{U}{dutchcal}{\skewchar\font=45 }
\DeclareFontShape{U}{dutchcal}{m}{n}{<-> s*[1.0] dutchcal-r}{}
\DeclareFontShape{U}{dutchcal}{b}{n}{<-> s*[1.0] dutchcal-b}{}
\DeclareMathAlphabet{\mathlcal}{U}{dutchcal}{m}{n}
\SetMathAlphabet{\mathlcal}{bold}{U}{dutchcal}{b}{n}

% mathscr font (supporting lowercase letters)
%\usepackage[scr=dutchcal]{mathalfa}
%\usepackage[scr=esstix]{mathalfa}
%\usepackage[scr=boondox]{mathalfa}
%\usepackage[scr=boondoxo]{mathalfa}
\usepackage[scr=boondoxupr]{mathalfa}
%\newcommand{\model}{\mathscr{h}}
\newcommand{\model}{\tilde{f}}
\newcommand{\rmod}{F}

\DeclareMathOperator{\cl}{cl}
\DeclareMathOperator{\vol}{vol}
\newcommand{\Set}[1]{\mathcal{#1}}
\DeclareMathAlphabet{\mathpzc}{OT1}{pzc}{m}{it} % The mathpzc font
\newcommand{\slv}{\mathpzc}
% mathpzc looks great, but it stops working on 19 Feb 2020 for no reason.
%\newcommand{\slv}{\mathscr}
\newcommand{\software}{\texttt}
\DeclareMathOperator{\eff}{\mathsf{e}\;\!}
\DeclareMathOperator{\Eff}{\mathsf{E}\;\!}
\newcommand{\out}{{\text{out}}}

\DeclareMathOperator{\comp}{C}
\DeclareMathOperator{\sign}{sign}


% Commands for revision
\newcommand{\REPHRASE}[1]{{\color{blue}{#1}}}
\newcommand{\TYPO}[1]{{\color{orange}{#1}}}
\newcommand{\MISTAKE}[1]{{\color{violet}{#1}}}
\newcommand{\REVISION}[1]{{\color{blue}{#1}}}
\newcommand{\REVISIONred}[1]{{\color{red}{#1}}}
%\newcommand{\COMMENT}[1]{\textcolor{darkgreen}{(#1)}}
%\newcommand{\REVISION}[1]{#1}
%\newcommand{\REVISIONred}[1]{#1}



%%%%%%%%%%%%%%%%%%%%%%%%%%%%%%%%%%%%%%%%%%%%%%%%%%%%%%%%%%%%%%%%%%%%%
\title{Notes on Singular Value Decomposition}
\date{February 20, 2020 (revision: \DTMnow)}
\author{
Z. Zhang
\thanks{
Department of Applied Mathematics, The Hong Kong Polytechnic University,
Hong Kong, China ({\tt zaikun.zhang@polyu.edu.hk}).
}
}

\begin{document}

\maketitle

\begin{abstract}
  We collect a few elementary facts about the singular value decomposition (SVD) of matrices. In
  particular, we present three approaches used by different authors in the history
  to establish the existence of~SVD.
\end{abstract}

\textbf{Keywords}.~Singular value decomposition, Eigenvalue decomposition, Deflation,
Jordan-Wielandt matrix

\textbf{Notation}.~Throughout this document, $m$, $n$, $k$, $i$, and~$j$ are positive integers; $r$ is a nonnegative integer.
We use $\|\cdot\|$ to denote the $2$-norm for vectors and matrices.
In inline equations, the MATLAB-style notation~$[a; b]$ stands for a vertical array
with~$a$ and~$b$ being its entries. The identity matrix is represented by~$I$, and it may take
a subscript to indicate its order when necessary. Given any matrix~$A$, $A_{i,j}$ signifies its~$(i,j)$ entry.


\section{Eigenvalue decomposition}

\begin{theorem}
  \label{th:evd}
  For any Hermitian matrix~$A\in \CC^{n\times n}$, there exists a unitary matrix~$U\in
  \CC^{n\times n}$ and a diagonal matrix~$\Lambda \in \RR^{n\times n}$ such that
  \begin{equation*}
    A \; = \; U\Lambda U^\hmt.
  \end{equation*}
  If~$A$ is real, then we can require that~$U$ is real.
  Indeed,~$\Lambda_{1,1}$, $\dots$, $\Lambda_{n,n}$ are the eigenvalues of~$A$, multiplicity
  included, and the~$j$-th column of~$U$ is an eigenvector of~$A$ associated with~$\Lambda_{j,j}$.
\end{theorem}

\begin{definition}
  \label{def:evd}
  Let~$A\in \CC^{n\times n}$ be an Hermitian matrix.
  \begin{enumerate}[leftmargin=1.5em]
    \item
      $U\Lambda U^\hmt$ is called an eigenvalue decomposition of~$A$, provided that~$A = U \Lambda U^\hmt$, $U
      \in \CC^{n\times n}$ is a unitary matrix, and~$\Lambda \in \RR^{n\times n}$ is a diagonal
      matrix.
    \item
      $U\Lambda U^\hmt$ is called a compact eigenvalue decomposition of~$A$,
      provided that~$A = U \Lambda U^\hmt$, $U \in \CC^{n\times r}$ is a matrix with~$U^\hmt U = I_r$,
      and~$\Lambda \in \RR^{r\times r}$ is a diagonal matrix whose diagonal entries are nonzero.
  \end{enumerate}
\end{definition}

%\begin{remark}
%  If~$U \Lambda U^\hmt$ is a compact eigenvalue decomposition of an Hermitian matrix~$A \in \CC^{n\times n}$,
%  then~$U$ has~$r=\rank(A)$ columns. Such a decomposition can be extended to an eigenvalue
%  decomposition
%  \begin{equation*}
%    (U \; \tilde{U})
%    \begin{pmatrix}
%      \Lambda & 0\\
%      0 & 0
%    \end{pmatrix}
%    \begin{pmatrix}
%      U^\hmt\\
%      \tilde{U}^\hmt
%    \end{pmatrix},
%  \end{equation*}
%  where~$\tilde{U}\in \CC^{n\times(n-r)}$ is any matrix such that~$(U \; \tilde{U})$ is unitary.
%\end{remark}


\section{Singular value decomposition (SVD)}

\begin{definition}
  \label{def:svd}
  Let~$A\in \CC^{m\times n}$ be a matrix with~$\rank(A) = r$.
  \begin{enumerate}[leftmargin=1.5em]
    \item
      $U\Sigma V^\hmt$ is called an singular value decomposition of~$A$,
      provided that~$A = U \Sigma V^\hmt$, $U
      \in \CC^{m\times m}$ and~$V\in\CC^{n\times n}$ are unitary matrices, and~$\Sigma \in
      \RR^{m\times n}$ is a matrix whose first~$r$ diagonal entries~(if~$r\ge 1$) are positive
      while all the other entries are zero.
    \item
      When~$r \ge 1$, $U\Sigma V^\hmt$ is called a compact (or \emph{reduced}) singular value decomposition of~$A$,
      provided that~$A = U \Sigma V^\hmt$, $U \in \CC^{m\times r}$ and~$V\in \CC^{n\times r}$
      satisfy~$U^\hmt U = V^\hmt V = I_r$,
      and~$\Sigma \in \RR^{r\times r}$ is a diagonal matrix whose diagonal entries are positive.
  \end{enumerate}
\end{definition}

\begin{remark}
  Let~$U\Sigma V^\hmt$ be a singular value decomposition of~$A\in \RR^{m\times n}$ and $\sigma_i
  = \Sigma_{i,i}$, where $1\le i\le \min\{m,n\}$. Then~\mbox{$\sigma_1$, $\dots$, $\sigma_r$} are called the
  \textnormal{(}nonzero\textnormal{)} singular values of~$A$. It is often convenient to
  regard $\sigma_{r+1}=\cdots=\sigma_{\min\{m,n\}} = 0$ also as singular values of~$A$.
\end{remark}

\begin{remark}
If~$U\Sigma V^\hmt$ is a \textnormal{(}compact\textnormal{)} singular value decomposition
of~$A$, then~$AV = U\Sigma$ and~$A^\hmt U = V\Sigma$.
Let~$\sigma_i=\Sigma_{i,i}$, $u_i$ be the~$i$-th column of~$U$,
and~$v_i$ be the~$i$-th column of~$V$.
Then~$A v_i = \sigma_i u_i$ and~$A^\hmt u_i = \sigma_i v_i$; $u_i$ and~$v_i$
are called a pair of left and right singular vectors of~$A$ associated with the
singular value~$\sigma_{i}$.
\end{remark}

\begin{remark}
  If~$U \Sigma V^\hmt$ is a compact singular value decomposition of~$A \in \CC^{m\times n}$,
  then we can extend it to a singular value decomposition
  \begin{equation*}
    (U \; \tilde{U})
    \begin{pmatrix}
      \Sigma & 0\\
      0 & 0
    \end{pmatrix}
    \begin{pmatrix}
      V^\hmt\\
      \tilde{V}^\hmt
    \end{pmatrix},
  \end{equation*}
  where~$\tilde{U}\in \CC^{m\times(m-r)}$ is any matrix such that~$(U \; \tilde{U})$ is unitary,
  and~$\tilde{V} \in \CC^{n\times (n-r)}$ is any matrix such that~$(V \; \tilde{V})$ is unitary.
  Conversely, if~$U\Sigma V^\hmt$ is a singular value decomposition of~$A\neq 0$, we can obtain a compact
  singular value decomposition by dropping the zero diagonal entries of~$\Sigma$ and the
  corresponding columns of~$U$ and~$V$.
\end{remark}


\subsection{Uniqueness of (compact) SVD}

\begin{theorem}
 Suppose that~$U\Sigma V^\hmt$ be a compact singular value decomposition of a matrix~$A$.
 \begin{enumerate}[leftmargin=1.5em]
   \item $U\Sigma^2 U^\hmt$ is a compact eigenvalue decomposition of~$AA^\hmt$,
     and hence the diagonal entries of~$\Sigma^2$ are the positive eigenvalues of~$AA^\hmt$, multiplicity included.
   \item $V\Sigma^2 V^\hmt$ is a compact eigenvalue decomposition of~$A^\hmt A$,
     and hence the diagonal entries of~$\Sigma^2$ are the positive eigenvalues of~$A^\hmt A$, multiplicity included.
 \end{enumerate}
\end{theorem}

\begin{lemma}
  \label{lem:commute}
  Consider matrices~$A \in \CC^{n\times n}$ and $\Lambda \in \CC^{n\times n}$ with~$\Lambda$ being diagonal.
  \begin{enumerate}[leftmargin=1.5em]
  \item $\Lambda A = A\Lambda$ if and only if~$A_{i,j} = 0$ for any~$i$ and~$j$ such that~$\Lambda_{i,i}\neq \Lambda_{j,j}$.
  \item If~$\Lambda$ is nonnegative and~$\Lambda A = A\Lambda$, then~$\Lambda^p A = A \Lambda^p$ for
    any~$p\ge 0$.
  \end{enumerate}
\end{lemma}

\begin{proof} Because
$\Lambda A = A \Lambda$ if and only if
$\Lambda_{i,i} A_{i,j} = A_{i,j} \Lambda_{j,j}$ for any $i,j\in\{1, \dots, n\}$.
\end{proof}

\begin{lemma}
  \label{lem:unitary}
  Let~$U_1$,~$U_2 \in \CC^{n\times r}$ satisfy~$U_1^\hmt U_1 = U_2^\hmt U_2
  = I_r$ and~$\range(U_1) = \range(U_2)$.
  \begin{enumerate}[leftmargin=1.5em]
    \item $U_1U_1^\hmt = U_2U_2^\hmt$, both being the orthogonal projection onto~$\range(U_1)
      = \range(U_2)$.
    \item $W=U_1^\hmt U_2$ is a unitary matrix and~$U_1W = U_2$.
  \end{enumerate}
\end{lemma}

\begin{proof}
  Since~$U_1^\hmt U_1 = I_r$, $U_1U_1^\hmt$ is the orthogonal projection onto~$\range(U_1)
  = \range(U_2)$ (see, \eg, \cite{Han_Neumann_2013}). In addition,
 $U_1 W = U_1 U_1^\hmt U_2 = U_2$, and $W^\hmt W = U_2^\hmt U_1 W = U_2^\hmt U_2 = I_r$.
\end{proof}


\begin{theorem}
  \label{th:svdunique}
  Let~$U_i\in \CC^{m\times r}$ and $V_i\in \CC^{n\times r}$ satisfy~$U_i^\hmt
  U_i=V_i^{\hmt} V_i = I_r$ \textnormal{(}$i=1,2$\textnormal{)}, and
  $\Sigma \in \CC^{r\times r}$ be a diagonal matrix whose diagonal entries are positive.
  Then~$U_1 \Sigma V_1^{\hmt} = U_2 \Sigma V_2^{\hmt}$ if and only if there exists
  a unitary matrix~$W\in\CC^{r\times r}$ such that~$U_2=U_1 W$, $V_2 = V_1 W$, and~$\Sigma
  W = W\Sigma$.
\end{theorem}

\begin{proof}
We will focus on the ``only if'' part since the~``if'' part is trivial.
Assuming that $U_1\Sigma V_1^\hmt = U_2\Sigma V_2^\hmt$,
we will show that~$W = U_1^\hmt U_2\in \CC^{r\times r}$ fulfills all the desired requirements.
Observe that both~$\Sigma V_1^\hmt$ and~$\Sigma V_2^\hmt$ have full column rank. Hence
\begin{equation*}
  \range(U_1) \;=\; \range(U_1\Sigma V_1^\hmt)
  \;=\; \range(U_2\Sigma V_2^\hmt) \;=\; \range(U_2).
\end{equation*}
By Lemma~\ref{lem:unitary},~$W$ is a unitary matrix and~$U_1W = U_2$.
It remains to show that~$\Sigma W = W\Sigma$ and~$V_2 = WV_1$.
Recalling that~$V_1^\hmt V_1 = V_2^\hmt V_2= I_r$, we have
\begin{equation}
  %\label{eq:us2u}
  \nonumber
  U_1\Sigma^2 U_1^\hmt
  \;=\; (U_1\Sigma V_1^\hmt) (U_1 \Sigma V_1^\hmt)^\hmt
  \;=\; (U_2\Sigma V_2^\hmt) (U_2 \Sigma V_2^\hmt)^\hmt
  \;=\; U_2\Sigma^2 U_2^\hmt.
\end{equation}
Hence
\begin{equation*}
  \Sigma^2W
  \;=\;\Sigma^2U_1^\hmt U_2
  \;=\; U_1^\hmt(U_1 \Sigma^2 U_1^\hmt) U_2
  \;=\; U_1^\hmt(U_2 \Sigma^2 U_2^\hmt) U_2
  \;=\; U_1^\hmt U_2 \Sigma^2
  \;=\; W\Sigma^2
\end{equation*}
Thus~$\Sigma W=W\Sigma$ by Lemma~\ref{lem:commute}.
Finally, since~$V_1 \Sigma U_1^\hmt =(U_1\Sigma V_1^\hmt)^\hmt = (U_2\Sigma V_2^\hmt)^\hmt= V_2 \Sigma
U_2^\hmt$,
\begin{equation*}
  V_2
  \;=\;(V_2\Sigma U_2^\hmt)(U_2 \Sigma^{-1})
  \;=\;(V_1\Sigma U_1^\hmt)(U_2 \Sigma^{-1})
  \;=\; V_1 \Sigma W \Sigma^{-1}
  \;=\; V_1 W \Sigma \Sigma^{-1}
  \;=\; V_1W.
\end{equation*}
The proof is complete.
\end{proof}


\subsection{Existence of SVD}

Here we present three independent ways of establishing the existence of
singular value decomposition (see~Theorems~\ref{th:svd}, \ref{th:ey}, \ref{th:lanczos}).
They were used by different authors in the history~\cite{Stewart_1993}.
Among these three approaches, Jordan's method does not depend on eigenvalue decomposition, but the
other two do.

Note that it suffices to prove the existence of the compact singular value decomposition, from which
a singular value decomposition can be constructed easily.

\subsubsection{Jordan's deflation approach~\cite{Jordan_1874}}

\begin{lemma}
  \label{lem:jordan}
  Given a nonzero matrix~$A\in\CC^{m\times n}$, let~$(u,v)\in \CC^{m}\times \CC^{n}$ be a solution of
  \begin{equation}
    \nonumber
    %\label{eq:jordan}
    \max\; \{\Re(x^\hmt A y) \mathrel{:} \|x\|=\|y\|=1, \; x \in \CC^{m}, \; y\in\CC^{n}\},
  \end{equation}
  and~${\sigma} = \Re(u^\hmt Av)$. Then~$Av = \sigma u$,~$A^\hmt u
  = \sigma v$, and~$\sigma>0$.
\end{lemma}

\begin{proof}
  Since~$A\neq 0$, it is obvious that~$\sigma >0$. Hence~$Av \neq 0$. According to the definition
  of~$u$,
  \begin{equation*}
    \Re(u^\hmt Av)\ge \Re(\left({Av}/{\|Av\|}\right)^\hmt Av) \;=\;
    \|Av\|\;=\;\|u\|\|Av\|.
  \end{equation*}
  By the Cauchy-Schwarz inequality, there exists a scalar~$\lambda>0$ such
  that~$\lambda u = Av$. Hence%
  \begin{equation*}
    \sigma \;=\; \Re(u^\hmt Av) \;=\; \Re(\lambda \|u\|^2) \;=\; \lambda.
  \end{equation*}
  Thus~$Av = \sigma u$. Similarly, we can prove~$A^\hmt u = \sigma v$ using the fact that
  \begin{equation*}
    \Re(u^\hmt A v)\;\ge\; \Re(u^\hmt A\left( A^\hmt u/\|A^\hmt
    u\|\right)) \;=\; \|A^\hmt u\| = \|A^\hmt u\|\|v\|.
  \end{equation*}
\end{proof}

%\begin{remark}
%  Indeed, the~$\sigma$ in Lemma~\ref{lem:jordan} is the largest singular value of~$A$, because
%  \begin{equation*}
%    \max_{\|x\|=\|y\|=1} \Re(x^\hmt A y) \;=\; \max_{\|y\|=1}\max_{\|x\|=1}\Re(x^\hmt Ay)
%    \;=\;\max_{\|y\|=1}\|Ay\| \;=\; \|A\|\;=\;\sigma_{\max}(A).
%  \end{equation*}
%  Similarly, we can see that
%  \begin{equation*}
%    \max_{\|x\|=\|y\|=1}|x^\hmt A y| \;=\; \sigma_{\max}(A).
%  \end{equation*}
%  See~\cite{Cao_Feng_2003} for more about variational representations for singular values of matrices.
%\end{remark}

\begin{remark}
  When Lemma~\ref{lem:jordan} is applied in the proof of Theorem~\ref{th:svd} later, we only need
  the existence of unit vectors~$u\in\CC^{m}$, $v\in\CC^{n}$, and a scalar~$\sigma>0$
  such that~$Av = \sigma u$ and~$A^\hmt u = \sigma v$. The existence can be established in other
  ways. Here are two examples.
  \begin{enumerate}[leftmargin=1.5em]
    \item Let~$\sigma = (\lambda_{\max}(AA^\hmt))^{\frac{1}{2}}$, $u\in\CC^{m\times m}$ be an eigenvector
      of~$AA^\hmt$ associated with~$\lambda_{\max}(AA^\hmt)$, and~$v\;=\;A^\hmt u/\sigma$.
      Then~$Av = AA^\hmt u/\sigma = \sigma^2 u/\sigma=\sigma u$, and~$A^\hmt u = \sigma v$.
      This is the approach used
      in the proofs of~\cite[Theorem~4.1]{Trefethen_Bau_1997} and~\cite[Theorem~1]{Koranyi_2001}.
    \item Let~$\sigma = \lambda_{\max}(J)$ with~$J$ being the Jordan-Wielandt form of~$A$
      \textnormal{(}see~\eqref{eq:JW}\textnormal{)}, $w\in\CC^{m+n}$ be an eigenvector associated with~$\sigma$,
      $x\in\CC^{m}$ consist of the first~$m$ entries of~$w$, and~$y\in\CC^{n}$ consist of the
      last~$n$. Then we can verify that~$Ay=\sigma x$ and~$A^\hmt x=\sigma y$.
      Meanwhile, $A(-y) = -\sigma x$, and~$A^\hmt x = -\sigma (-y)$, making~$[x; -y]$
      an eigenvector of~$J$ associated with~$-\sigma \neq \sigma$. Since~$J$ is Hermitian,
      we know that~$w$ and~$[x; -y]$ are orthogonal, and hence~$x^\hmt x - y^\hmt y= 0$. Thus~$\|x\|=\|y\|$,
    which are nonzero since~$w\neq 0$. Finally, let~$u = x/\|x\|$ and~$v = y/\|y\|$.
  \end{enumerate}
\end{remark}

\begin{theorem}
  \label{th:svd}
  Any~$A\in\CC^{m\times n}$ has a singular value decomposition~$U\Sigma V^\hmt$
  as defined in Definition~\ref{def:svd}.
\end{theorem}

\begin{proof}
  Assume without loss of generality that~$A\neq 0$. We prove by an induction on~$\min\{m,n\}$.

  1.~If~$\min\{m, n\} = 1$, then~$A$ is either a row or a column. If~$A$ is a column, let
  $U$ be a unitary matrix whose first column is~$A/\|A\|$, $\Sigma = e_1$ (\ie, the first canonical
  coordinate vector), and~$V = \|A\|$. Then~$U\Sigma V^\hmt$ is a singular value decomposition
  of~$A$. If~$A$ is a row, the decomposition can be found similarly.

  2.~Assume that the conclusion holds when~$\min\{m,n\}= k$. Let us consider the scenario where
  $\min\{m,n\} = k+1$. Let~$A$ be a matrix in~$\CC^{m\times n}$.
  %Without loss of generality, suppose that~$A\neq0$.
  By Lemma~\ref{lem:jordan},
  there exist unit vectors~$u\in\CC^{m}$, $v\in\CC^{n}$, and a scalar~$\sigma>0$ such that
  \begin{equation}
    \label{eq:jordan}
    Av \;=\; \sigma u, \quad A^\hmt u\;=\;\sigma v.
  \end{equation}
  Let~$U\in \CC^{m\times m}$ be a unitary matrix whose first column is~$u$, and~$V\in\CC^{n\times n}$
  be a unitary matrix whose first column is~$v$. It is then straightforward to check that
  \begin{equation}
    \label{eq:aind1}
    U^\hmt A V \;=\;
    \begin{pmatrix}
      \sigma & 0\\
      0 & \hat{A}
    \end{pmatrix},
  \end{equation}
  where~$\hat{A}$ is a matrix in~$\CC^{(m-1)\times (n-1)}$. If~$\hat{A} = 0$, then~\eqref{eq:aind1}
  provides a singular value decomposition for~$A$.
  Otherwise, since~$\min\{m-1, n-1\} = \min\{m,n\}-1$, we
  know from the induction hypothesis that~$\hat{A}$ has a singular value
  decomposition~$\hat{U}\hat{\Sigma} \hat{V}^\hmt$. Consequently,
  \begin{equation}
    \label{eq:aind2}
    A \;=\; U (U^\hmt A V)V^\hmt
    \;=\;
    U
    \begin{pmatrix}
    \sigma & 0\\
    0 & \hat{U}\hat{\Sigma} \hat{V}^\hmt
    \end{pmatrix}
    V^\hmt
  \;=\;
    \left[
  U
  \begin{pmatrix}
    1 & 0\\
    0 & \hat{U}
  \end{pmatrix}
\right]
  \begin{pmatrix}
    \sigma & 0\\
    0 & \hat{\Sigma}
  \end{pmatrix}
  \left[
  \begin{pmatrix}
    1 & 0\\
    0 & \hat{V}
  \end{pmatrix}^{\!\hmt}
  \,V^\hmt
\right].
  \end{equation}
  It is easy to verify that the right-hand side of~\eqref{eq:aind2} provides a singular value
  decomposition for~$A$. This completes the induction.
\end{proof}

\begin{remark}
  We can also prove Theorem~\ref{th:svd} by an induction on~$\rank(A)$. When~$\rank(A) \!= 0$,
  the desired conclusion is trivial. Assume that the conclusion holds when~$\rank(A) \le k$. Let us consider
  the scenario with~$\rank(A) = k+1$. By Lemma~\ref{lem:jordan}, there exists unit
  vectors~$u\in\CC^{m}$, $v\in\CC^{n}$, and a scalar ~$\sigma>0$ fulfilling~\eqref{eq:jordan}.
  Define~$\hat{A} = A-\sigma uv^\hmt$. Then it is easy to check that~$\ker(A)\subset
  \ker(\hat{A})$ and~$v\in\ker(\hat{A})$. Since~$v\in\range(A^\hmt)\perp \ker(A)$, we know
  that~$\dim\ker(\hat{A})\ge \dim\ker(A)+1$. Thus~$\rank(\hat{A}) \le \rank(A)-1$.
  If~$\hat{A} = 0$, we are done. Otherwise, by the induction
  hypothesis,~$\hat{A}$ has a compact singular value decomposition~$\hat{U}\hat{\Sigma}\hat{V}^\hmt$.
  Consequently,
  \begin{equation}
    \label{eq:aind3}
    A\;=\; \sigma uv^\hmt + \hat{A} \;=\;
    \sigma uv^\hmt + \hat{U}\hat{\Sigma} \hat{V}^\hmt \;=\;
      (u\,\; \hat{U})
    \begin{pmatrix}
      \sigma & 0\\
      0 &\hat{\Sigma}
    \end{pmatrix}
    (v\,\; \hat{V})^\hmt.
  \end{equation}
  Noting that~$\hat{A}v = 0$, $\hat{A}^\hmt u = 0$, and~$\hat{\Sigma}$ is nonsingular, we can see
  that~$\hat{V}^\hmt v = 0$ and~$\hat{U}^\hmt u  = 0$. Thus the columns of~$(u\,\;\hat{U})$ are
  orthonormal, and so are those of~$(v\,\; \hat{V})$. Hence~\eqref{eq:aind3} provides a compact
  singular value decomposition for~$A$, which can be extended to a singular value decomposition. The
  induction is complete.
\end{remark}


\subsubsection{The Eckart-Young approach~\cite{Eckart_Young_1939}}

\begin{lemma}
  \label{lem:fro2}
  Suppose that~$A \in \CC^{m\times n}$ and~$B \in \CC^{n\times k}$.
  \begin{enumerate}[leftmargin=1.5em]
    \item $\|AB\|_\fro \le \|A\| \|B\|_\fro$, and the equality holds if and only if~$A^\hmt A B = \|A\|^2 B$.
    \item $\|AB\|_\fro \le \|A\|_\fro\|B\|$, and the equality holds if and only if~$ABB^\hmt = \|B\|^2 A$.
  \end{enumerate}
\end{lemma}

\begin{remark}
  Recall that~$\|\cdot\|$ denotes the~$2$-norm for matrices.
\end{remark}

\begin{proof}
  Let~$C\in\CC^{n\times n}=(\|A\|^2 I - A^\hmt A)^\frac{1}{2}$, which is well defined since~$\|A\|^2
  I - A^\hmt A$ is positive semidefinite.~Then%
  \begin{equation*}
  \|A\|^2\|B\|_\fro^2 - \|AB\|_\fro^2 \;=\; \tr(\|A\|^2 B^\hmt B) - \tr(B^\hmt A^\hmt A B)\;=\; \tr(B^\hmt C^2 B)\;\ge\; 0.
  \end{equation*}
  Thus~$\|AB\|_\fro \le \|A\|\|B\|_\fro$, and
  \begin{equation*}
    \|AB\|_\fro = \|A\|\|B\|_\fro
    \;\Longleftrightarrow \; B^\hmt C^2 B =0
    \;\Longleftrightarrow \; C^2 B =0
    \;\Longleftrightarrow \; A^\hmt A B = \|A\|^2 B.
  \end{equation*}
  The proof concerning~$\|AB\|_\fro \le \|A\|_\fro\|B\|$ is similar.
\end{proof}

\begin{lemma}
  \label{lem:uav}
  Let~$A$, $B$, $U$, and~$V$ be complex matrices such that both~$U^\hmt AV$
  and~$UBV^\hmt$ are well defined.
  If~$\|A\|_\fro=\|B\|_\fro$ and~$\|U\|\|V\|=1$, then $A = UBV^\hmt$ if and only if
  $B = U^\hmt A V$.
\end{lemma}

\begin{proof}
  Without loss of generality, we suppose that~$\|U\|=\|V\|=1$. Otherwise, consider~$U/\|U\|$
  and~$V/\|V\|$ instead of~$U$ and~$V$ respectively.

  Assume that~$A = UBV^\hmt$. Since~$\|U\|=\|V\|=1$ and~$\|A\|_\fro=\|B\|_\fro$, we have
 \begin{equation*}
   \min\{\|UB\|_\fro,\; \|BV^\hmt\|_\fro\}\;\ge\;\|UBV^\hmt\|_\fro\;=\;\|A\|_\fro = \|B\|_\fro.
 \end{equation*}
 Hence Lemma~\ref{lem:fro2} ensures
 \begin{equation*}
   U^\hmt U B \;=\; B, \quad BV^\hmt V \;=\; B.
 \end{equation*}
 Therefore,
 \begin{equation*}
   U^\hmt A V \;=\; U^\hmt U B V^\hmt V \;=\; B V^\hmt V \;=\; B.
 \end{equation*}
 In the same way, $B=U^\hmt A V$ implies~$A = UBV^\hmt$.
\end{proof}


\begin{theorem}
  \label{th:ey}
  Let~$A \in \CC^{m\times n}$ be a matrix.
  \begin{enumerate}[leftmargin=1.5em]
  \item If~$V\Lambda V^\hmt$ is a compact eigenvalue decomposition of~$A^\hmt A$ and~$U = AV\Lambda^{-\frac{1}{2}}$,
    then~$U\Lambda^\frac{1}{2} V^\hmt$ is a compact singular value decomposition of~$A$.
  \item If~$U\Lambda U^\hmt$ is a compact eigenvalue decomposition of~$AA^\hmt$ and~$V = A^\hmt U\Lambda^{-\frac{1}{2}}$,
    then~$U\Lambda^\frac{1}{2} V^\hmt$ is a compact singular value decomposition of~$A$.
  \end{enumerate}
\end{theorem}

\begin{proof}
  We only prove 1. By assumption,~$V^\hmt V=I$,~$A^\hmt A = V\Lambda V^\hmt$,
  and~$U = A V\Lambda^{-\frac{1}{2}}$. Hence
  \begin{equation}
    \label{eq:uhu}
    U^\hmt U \;=\;  (A V\Lambda^{-\frac{1}{2}})^\hmt  (AV\Lambda^{-\frac{1}{2}})
    \;=\; \Lambda^{-\frac{1}{2}}(V^\hmt A^\hmt A V)\Lambda^{-\frac{1}{2}}
    \;=\; \Lambda^{-\frac{1}{2}}\Lambda\Lambda^{-\frac{1}{2}} \;=\; I.
  \end{equation}
  Thus
  \begin{equation*}
    U^\hmt A V \;=\; U^\hmt (U\Lambda^{\frac{1}{2}}) \;=\; \Lambda^{\frac{1}{2}}.
  \end{equation*}
  Meanwhile, $\|U\|=1$ by~\eqref{eq:uhu}, $\|V\|=1$ because~$V^\hmt V = I$, and
  \begin{equation*}
    \|A\|_\fro^2 \;=\; \tr(A^\hmt A) \;=\; \tr(\Lambda) \;=\; \|\Lambda^{\frac{1}{2}}\|_\fro^2.
  \end{equation*}
  Therefore, Lemma~\ref{lem:uav} ensures
  \begin{equation}
    \nonumber
    %\label{eq:Adec}
    A \;=\; U\Lambda^{\frac{1}{2}}V^\hmt.
  \end{equation}
  Hence~$U\Lambda^{\frac{1}{2}}V^\hmt$ is a compact singular value decomposition of~$A$.
\end{proof}

\begin{remark}
The major point of the proof is to show that~$U\Lambda^{\frac{1}{2}}V^\hmt =A$. Here we use
Lemma~\ref{lem:uav}, but there are other ways to prove it.
\end{remark}

\begin{corollary}
  \label{coro:eyfull}
  Let~$A \in \CC^{m\times n}$ be a matrix.
  \begin{enumerate}[leftmargin=1.5em]
    \item If~$V\Lambda V^\hmt$ is an eigenvalue decomposition of~$A^\hmt A$ such
      that the diagonal entries of~$\Lambda$ are descending. The there exist~$U\in \CC^{m\times m}$
      and~$\Sigma \in \RR^{m\times n}$ such that~$U\Sigma V^\hmt$ is a singular value decomposition of~$A$.
    \item If~$U\Lambda U^\hmt$ is an eigenvalue decomposition of~$AA^\hmt$ such
      that the diagonal entries of~$\Lambda$ are descending. The there exist~$V\in \CC^{n\times n}$
      and~$\Sigma \in \RR^{m\times n}$ such that~$U\Sigma V^\hmt$ is a singular value decomposition of~$A$.
  \end{enumerate}
\end{corollary}

\begin{proof}
  We only prove 1. Suppose that~$\rank(A) = r$. Let~$\hat{\Lambda} = \diag(\Lambda_{1,1}, \dots,
  \Lambda_{r,r})$ and~$\hat{V}$ be the first~$r$ columns of~$V$. Then~$\hat{V} \hat{\Lambda}
  \hat{V}^\hmt$ is a compact eigenvalue decomposition of~$A^\hmt A$. With~$\hat{U}
  = A\hat{V}\hat{\Lambda}^{-\frac{1}{2}}$, we know that~$\hat{U}\hat{\Lambda}^{\frac{1}{2}}\hat{V}^\hmt$ is
  a compact singular value decomposition of~$A$.
  Let~$\tilde{U}\in \CC^{m\times(m-r)}$ be any matrix such that~$(\hat{U}\; \tilde{U})$ is
  unitary. Then
  \begin{equation*}
    (\hat{U}\; \tilde{U})
    \begin{pmatrix}
      \hat{\Lambda}^{\frac{1}{2}} & 0\\
      0 & 0
    \end{pmatrix}
    V^\hmt
  \end{equation*}
  is a singular value decomposition of~$A$.
\end{proof}

\subsubsection{The Wielandt-Lanczos approach~\cite{Lanczos_1958}}

\begin{lemma}
  \label{lem:lanczos}
  Given a matrix~$A \in \CC^{m\times n}$, define its Jordan-Wielandt form~\cite{Mathias_2013} to be
\begin{equation}
  \label{eq:JW}
  J =
  \begin{pmatrix}
    0 & A \\
    A^\hmt & 0
  \end{pmatrix}.
\end{equation}
 Then the characteristic polynomial of~$J$ is
    \begin{equation}
      \label{eq:chp}
      p(\sigma) \;=\; \sigma^{m-n}\det(\sigma^2 I_n - A^\hmt A) \;=\; \sigma^{n-m}\det(\sigma^2 I_m
      - A A^\hmt).
    \end{equation}
    If the nonzero eigenvalues of~$AA^\hmt$ \textnormal{(}i.e., those of~$A^\hmt A$\textnormal{)}
    are~$\lambda_1$, $\dots$, $\lambda_r$, multiplicity included, then
    the nonzero eigenvalues of~$J$ are~$\sqrt{\lambda_1}$, $-\sqrt{\lambda_1}$, $\dots$, $\sqrt{\lambda_r}$,
    $-\sqrt{\lambda_r}$, multiplicity included.
%\begin{enumerate}
%  \item
%    The characteristic polynomial of~$J$ is
%    \begin{equation}
%      \label{eq:eig}
%      p(\sigma) \;=\; \sigma^{m-n}\det(\sigma^2 I_n - A^\hmt A) \;=\; \sigma^{n-m}\det(\sigma^2 I_m
%      - A A^\hmt).
%    \end{equation}
%    If the nonzero eigenvalues of~$AA^\hmt$ (i.e., those of~$A^\hmt A$) are~$\lambda_1 \ge \cdots \ge \lambda_r > 0$, then
%    the nonzero eigenvalues of~$J$ are~$\sqrt{\lambda_1}$, $-\sqrt{\lambda_1}$, $\dots$, $\sqrt{\lambda_r}$,
%    $-\sqrt{\lambda_r}$, multiplicity included.
%  \item The eigenspace of~$J$ associated with an eigenvalue~$\sigma$ is
%    \begin{equation}
%      \label{eq:eigs}
%      \{[u; v]\in\CC^{m+n} \mathrel{:} A^\hmt u = \sigma v, Av = \sigma u, u\in \CC^{m}, v\in
%    \CC^n\}.
%    \end{equation}
%     If~$\sigma \neq 0$, then the eigenspace can be further specified as
%    \begin{equation*}
%      \left\{ [u; A^\hmt u/\sigma] \mathrel{:} A A^\hmt u = \sigma^2 u, u\in \CC^{m} \right\}
%      \;=\;
%      \left\{ [Av/\sigma; v] \mathrel{:} A^\hmt A v = \sigma^2 v, v\in \CC^{n} \right\};
%    \end{equation*}
%    if~$\sigma = 0$, then it is
%    \begin{equation*}
%      \left\{ [u; v] \mathrel{:} A^\hmt u  = 0, A v = 0 \right\} \;=\; \ker(A^\hmt) \times \ker(A).
%    \end{equation*}
%  \item If~$U\Lambda U^\hmt$ is a compact eigenvalue decomposition of~$AA^\hmt$ and~$V = A^\hmt U\Lambda^{-\frac{1}{2}}$,
%    or~$V\Lambda V^\hmt$ is a compact eigenvalue decomposition of~$A^\hmt A$ and~$U = A U\Lambda^{-\frac{1}{2}}$,
%    then
%    \begin{equation*}
%      \left[
%      \frac{1}{\sqrt{2}}
%      \begin{pmatrix}
%        U & -U\\
%        V & V
%      \end{pmatrix}
%    \right]
%      \begin{pmatrix}
%        \Lambda^{\frac{1}{2}} & 0\\
%        0 & -\Lambda^{\frac{1}{2}}
%      \end{pmatrix}
%      \left[
%      \frac{1}{\sqrt{2}}
%      \begin{pmatrix}
%        U^\hmt & V^\hmt\\
%        -U^\hmt & V^\hmt
%      \end{pmatrix}
%    \right]
%    \end{equation*}
%    is a compact eigenvalue decomposition of~$J$. Consequently, $U\Lambda^{\frac{1}{2}} V^\hmt$ is
%    a compact singular value decomposition of~$A$.
%\end{enumerate}
\end{lemma}

\begin{proof}
  We only prove the first equality in~\eqref{eq:chp}. For any~$\sigma \neq 0$,
  \begin{equation*}
    \begin{pmatrix}
      I_m & 0 \\
      \sigma^{-1}A^\hmt & I_n
    \end{pmatrix}
    \begin{pmatrix}
      \sigma I_m & -A \\
      -A^\hmt & \sigma I_n
    \end{pmatrix}
    =
    \begin{pmatrix}
      \sigma I_m & -A\\
      0 & \sigma I_n - \sigma^{-1}A^\hmt A
    \end{pmatrix}.
  \end{equation*}
  Taking determinants, we have
  \begin{equation}
    \label{eq:rational}
    \det(\sigma I - J) \;=\; \det(\sigma I_m) \det(\sigma I_n - \sigma^{-1}A^\hmt A) \;=\;
    \sigma^{m-n} \det(\sigma^2 I_n -A^\hmt A).
  \end{equation}
  In~\eqref{eq:rational}, two rational functions are equal for all~$\sigma \neq 0$.
  Hence they are indeed identical.
\end{proof}


\begin{theorem}
  \label{th:lanczos}
  Consider matrices~$A\in \CC^{m\times n}$, $\Sigma \in \RR^{r\times r}$, $U_i \in
  \CC^{m\times r}$, and $V_i \in \CC^{n\times r}$ \textnormal{(}$i=1,2$\textnormal{)}.
  Suppose that~$\Sigma$ is a diagonal matrix
  whose diagonal entries are positive. Then
  \begin{equation}
    \label{eq:evdJ}
    \begin{pmatrix}
      U_1 & U_2 \\
      V_1 & V_2
    \end{pmatrix}
    \begin{pmatrix}
      \Sigma & 0\\
      0 & -\Sigma
    \end{pmatrix}
    \begin{pmatrix}
      U_1 & U_2 \\
      V_1 & V_2
    \end{pmatrix}^\hmt
  \end{equation}
  is a compact eigenvalue decomposition of the Jordan-Wielandt matrix~$J$ in~\eqref{eq:JW}
  if and only if both~$(\sqrt{2} U_1)\Sigma (\sqrt{2} V_1)^\hmt$ and~$(-\sqrt{2}U_2)\Sigma (\sqrt{2}V_2)^\hmt$ are
  compact singular value decompositions of~$A$.
\end{theorem}

\begin{proof} 1.~Assume that~\eqref{eq:evdJ} is a compact eigenvalue decomposition of~$J$.
  We will prove that $(\sqrt{2} U_1)\Sigma (\sqrt{2} V_1)^\hmt$ is a
  compact singular value decompositions~of~$A$, and the other one can be discussed similarly.
  It suffices to show that
  \begin{equation}
    \label{eq:svdA}
    U_1^\hmt U_1 \;=\; V_1^\hmt V_1 \;=\; \dfrac{I_r}{2},
    \quad U_1\Sigma V_1^\hmt \;=\; \dfrac{A}{2}.
    %\quad  U_2^\hmt U_2 \;=\; V_2^\hmt V_2 \;=\; \dfrac{I_r}{2},
    %\quad -U_2\Sigma V_2^\hmt \;=\; \dfrac{A}{2}.
  \end{equation}
  Due to the compact eigenvalue decomposition~\eqref{eq:evdJ} of~$J$,
  the columns of~$[U_1; V_1]$ are eigenvectors of~$J$ associated with
  all its~$r$ positive eigenvalues,\footnote{
   Recall that the MATLAB-style notation~$[a; b]$ denotes a vertical
   array with~$a$ and~$b$ being its entries.
  }
  and
  \begin{equation}
    \label{eq:evd1}
    J
    \begin{pmatrix}
      U_1 \\
      V_1
    \end{pmatrix}
    =
    \begin{pmatrix}
      U_1 \\
      V_1
    \end{pmatrix}
    \Sigma,
    %\quad U_1^\hmt U_1 + V_1^\hmt V_1 = I_r,
  \end{equation}
   This implies $AV_1 = U_1\Sigma$ and~$A^\hmt U_1 = \Sigma V_1$, which can be reformulated as
  \begin{equation}
    \label{eq:evd2}
    J
    \begin{pmatrix}
      U_1 \\
      -V_1
    \end{pmatrix}
    =
    \begin{pmatrix}
      U_1 \\
      -V_1
    \end{pmatrix}
    (-\Sigma),
    %\quad U_1^\hmt U_1 + (-V_1)^\hmt (-V_1) = I_r.
  \end{equation}
  \ie, the columns of~$[U_1; -V_1]$ are eigenvectors of~$J$ associated with the negative eigenvalues.
  Hence the columns of~$[U_1; V_1]$ and those of~$[U_1; -V_1]$ are orthogonal. Thus
  \begin{equation*}
    U_1^\hmt U_1 -V_1^\hmt V_1 \;=\;
    (U_1^\hmt \;\, V_1^\hmt)
    \begin{pmatrix}
      U_1\\
      -V_1
    \end{pmatrix} \;=\;0.
  \end{equation*}
  With~\eqref{eq:evdJ} being a compact eigenvalue decomposition, we also have~$U_1^\hmt U_1 + V_1^\hmt V_1 = I_r$.
  Hence $U_1^\hmt U_1 = V_1^\hmt V_1= {I_r}/{2}$,
  which is the first equality in~\eqref{eq:svdA}. To establish the second one, define
  \begin{equation*}
    \bar{U}\;=\;
    \begin{pmatrix}
      U_1 & U_1 \\
      V_1 & -V_1
    \end{pmatrix},
    \quad
      \bar{\Sigma} \;=\;
    \begin{pmatrix}
        \Sigma & 0\\
        0&-\Sigma
    \end{pmatrix}.
  \end{equation*}
  Then
  \begin{equation}
    \label{eq:zortho}
    \bar{U}^\hmt \bar{U} \;=\;
    \begin{pmatrix}
      U_1^\hmt U_1 + V_1^\hmt V_1 & U_1^\hmt U_1-V_1^\hmt V_1\\
      U_1^\hmt U_1 - V_1^\hmt V_1 & U_1^\hmt U_1+V_1^\hmt V_1
    \end{pmatrix}
    \;=\;
    \begin{pmatrix}
      I_r & 0\\
      0 & I_r
    \end{pmatrix}.
  \end{equation}
  Meanwhile, we can reformulate~\eqref{eq:evd1}--\eqref{eq:evd2} as $J\bar{U}= \bar{U}\bar{\Sigma}$.
  Therefore,
  \[
  \bar{U}^\hmt J \bar{U} \;=\; \bar{\Sigma}.
  \]
  Note that~$\|J\|_\fro = \|\bar{\Sigma}\|_\fro$ according to the compact eigenvalue
  decomposition~\eqref{eq:evdJ}
  and $\|\bar{U}\|=1$ due to~\eqref{eq:zortho}.
  Thus Lemma~\ref{lem:uav} renders
    \begin{equation}
      \label{eq:jdec}
      J \;=\; \bar{U}\bar{\Sigma}\bar{U}^\hmt,
    \end{equation}
  from which we can obtain $A = 2U_1\Sigma V_1^\hmt$ by straightforward calculations.

  2.~Assume that both~$(\sqrt{2}U_1)\Sigma (\sqrt{2}V_1)^\hmt$
  and~$(-\sqrt{2}U_2)\Sigma(\sqrt{2}V_2)^\hmt$ are compact singular value decompositions of~$A$.
  Then we have~\eqref{eq:svdA} and
  \begin{equation}
    \label{eq:svdA2}
     U_2^\hmt U_2 \;=\; V_2^\hmt V_2 \;=\; \dfrac{I_r}{2},
    \quad -U_2\Sigma V_2^\hmt \;=\; \dfrac{A}{2}.
  \end{equation}
  To prove that~\eqref{eq:evdJ} is a compact singular value decomposition for~$J$, it suffices to show
  \begin{equation*}
    \begin{pmatrix}
      0 & A\\
      A^\hmt & 0
    \end{pmatrix}
    =
    \begin{pmatrix}
      U_1 & U_2\\
      V_1 & V_2
    \end{pmatrix}
    \begin{pmatrix}
      \Sigma & 0\\
      0& -\Sigma
    \end{pmatrix}
    \begin{pmatrix}
      U_1 & U_2\\
      V_1 & V_2
    \end{pmatrix}^\hmt,
    \quad\;
  %\end{equation*}
  %\begin{equation*}
    \begin{pmatrix}
      U_1 & U_2\\
      V_1 & V_2
    \end{pmatrix}^\hmt
    \begin{pmatrix}
      U_1 & U_2\\
      V_1 & V_2
    \end{pmatrix}
    =
    \begin{pmatrix}
      I_r & 0\\
      0 & I_r
    \end{pmatrix},
  \end{equation*}
  which resolve to
  \begin{empheq}[left=\empheqlbrace]{align}
    \label{eq:Jdec}&U_1\Sigma U_1^\hmt - U_2\Sigma U_2^\hmt \;=\; 0, \quad
    V_1\Sigma V_1^\hmt - V_2\Sigma V_2^\hmt \;=\; 0, \quad
    U_1\Sigma V_1^\hmt - U_2\Sigma V_2^\hmt \;=\; A,\\[1ex]
    \label{eq:Uorth}&U_1^\hmt U_1 + V_1^\hmt V_1 \;=\; I_r, \quad
    U_2^\hmt U_2 + V_2^\hmt V_2 \;=\; I_r, \quad
    U_1^\hmt U_2 + V_1^\hmt V_2 \;=\; 0.
  \end{empheq}
  Since~$U_1\Sigma V_1^\hmt = -U_2\Sigma V_2^\hmt = A/2$, Theorem~\ref{th:svdunique} implies
  the existence of a unitary matrix~$W\in \CC^{r\times r}$ such that
  \begin{equation*}
    U_2 \;=\; -U_1W, \quad V_2 \;=\; V_1W, \quad \Sigma W \;=\; W\Sigma.
  \end{equation*}
  Hence
  \begin{equation*}
    U_2\Sigma U_2^\hmt \;=\; (-U_1W)\Sigma (-U_1W)^\hmt \;=\; U_1W\Sigma W^\hmt U_1^\hmt \;=\; U_1 \Sigma
    WW^\hmt U_1^\hmt \;=\; U_1\Sigma U_1^\hmt,
  \end{equation*}
  which implies~$U_1\Sigma U_1^\hmt - U_2\Sigma U_2^\hmt = 0$. Similarly, $V_1\Sigma V_1^\hmt
  - V_2\Sigma V_2^\hmt = 0$. In addition,
  \begin{equation*}
    U_1^\hmt U_2 + V_1^\hmt V_2 \;=\; -U_1^\hmt U_1W + V_1^\hmt V_1W \;=\; 0,
  \end{equation*}
  where we use the fact that~$U_1^\hmt U_1 = V_1^\hmt V_1 = I_r/2$ from~\eqref{eq:svdA}.
  By~\eqref{eq:svdA} and~\eqref{eq:svdA2}, we also have
  \begin{equation*}
    U_1^\hmt U_1 + V_1^\hmt V_1 \;=\;  U_2^\hmt U_2 + V_2^\hmt V_2 \;=\; I_r, \quad U_1\Sigma V_1^\hmt
    - U_2\Sigma V_2^\hmt \;=\; A.
  \end{equation*}
  All the equalities in~\eqref{eq:Jdec}--\eqref{eq:Uorth} have been justified. The proof is
  complete.
\end{proof}

\begin{remark}
  According to~\eqref{eq:zortho} and~\eqref{eq:jdec},
  $\bar{U}\bar{\Sigma}\bar{U}^\hmt$ is indeed a compact eigenvalue decomposition of~$J$.
\end{remark}


\section{Decompose a matrix into partial isometries~\cite{Koranyi_2001}}

\begin{definition}
  A matrix~$A\in\CC^{m\times n}$ is said to be a partial isometry if~$\|Ax\|=\|x\|$ for
  each~$x\in \range(A^\hmt)$ (\ie,~$x\in (\ker A)^{\perp}$).
\end{definition}

The following proposition collects various characterizations of partial isometries.

\begin{proposition}
  \label{prop:piso}
  For any~$A\in\CC^{m\times n}$, the following statements are equivalent.
  \begin{enumerate}[leftmargin=1.5em]
    \item \label{it:ai} $A$ is a partial isometry.
    \item \label{it:ahi} $A^\hmt$ is a partial isometry.
    \item \label{it:aha} $A^\hmt A$ is an orthogonal projection.
    \item \label{it:aah} $AA^\hmt$ is an orthogonal projection.
    \item \label{it:ahaah} $A^\hmt A A^\hmt = A^\hmt$.
    \item \label{it:aaha} $A A^\hmt A = A$.
    \item \label{it:sina} All the nonzero singular values of~$A$ are~$1$.
    \item \label{it:iso} The linear operator~$T\mathrel{:} x \mapsto Ax$ is an isometric isomorphism from~$\range(A^\hmt)$ to $\range(A)$.
  \end{enumerate}
\end{proposition}

\begin{proof}
  \ref{it:ai} $\Rightarrow $ \ref{it:iso}.~Obvious.

  \ref{it:iso} $\Rightarrow $ \ref{it:ahi}.~Take any~$x\in \range(A)$.
  There is a~$y \in \range(A^\hmt)$ such that~$x=Ay$. By assumption, $\|y\|=\|x\|$. Hence
    \begin{equation*}
      \|A^\hmt x\| \;\ge\; (y/\|y\|)^\hmt A^\hmt x \;=\; \frac{1}{\|x\|} y^\hmt A^\hmt x \;=\;
      \frac{1}{\|x\|} x^\hmt x\;=\; \|x\|.
    \end{equation*}
  For any unit vector~$z\in\CC^{n}$, let~$z'$ be its orthogonal projection on~$\range(A^\hmt)$.
  Then
  \begin{equation*}
  \|Az\|\;=\;\|Az'\|\;=\;\|z'\|\;\le\;\|z\|\;=\;1.
  \end{equation*}
  Thus
    \begin{equation*}
      \|A^\hmt x\| \;=\; \max_{\|z\|=1}z^\hmt A^\hmt x
      \;\le\; \max_{\|z\|=1}\|Az\|\|x\| \;\le\; \|x\|.
    \end{equation*}

    \ref{it:ahi} $\Rightarrow $ \ref{it:aha}.~Since~$A^\hmt A$ is Hermitian, it suffices to show that it is idempotent. We
      only need to prove that~$x^\hmt(A^\hmt A)^2y=x^\hmt A^\hmt Ay$ for any~$x$ and~$y\in \RR^n$,
      or equivalently, $u^\hmt A A^\hmt v=u^\hmt v$ for  any~$u$ and~$v\in \range(A)$. By
      assumption,
      For any~$u$, $v\in\range(A)$, we have $\|A^\hmt u\|=\|u\|$, $\|A^\hmt v\|=\|v\|$, and
      $\|A^\hmt (u+v)\| = \|u+v\|$. Squaring the last equality, we obtain~$u^\hmt AA^\hmt v = u^\hmt
      v$.

      \ref{it:aha}~$\Rightarrow$~\ref{it:ahaah}.~Since~$A^\hmt A$ is a projection, we have~$A^\hmt A x = x$ for all~$x\in \range(A^\hmt A)
    = \range(A^\hmt)$. Hence~$A^\hmt AA^\hmt = A^\hmt$.

    \ref{it:ahaah}~$\Rightarrow$~\ref{it:aah}.~$AA^\hmt$ is Hermitian, and~$(AA^\hmt)^2=AA^\hmt AA^\hmt = AA^\hmt$.

    \ref{it:aah}~$\Rightarrow$~\ref{it:aaha}.~Similar to~\ref{it:aha} $\Rightarrow$ \ref{it:ahaah}.

    \ref{it:aaha}~$\Rightarrow$~\ref{it:sina}.~Since~$AA^\hmt$ is positive semidefinite
    and~$(AA^\hmt)^2=AA^\hmt AA^\hmt =AA^\hmt$, we know that all the nonzero eigenvalues of~$AA^\hmt$ are $1$.

    \ref{it:sina}~$\Rightarrow$~\ref{it:ai}.~Let~$r=\rank(A)$. If~$r = 0$, then the conclusion is trivially true.
       Otherwise, $A$ has a compact singular value decomposition of the form~$UV^\hmt$,
       where~$U\in\CC^{m\times r}$ and~$V\in\CC^{n\times r}$ satisfy~$U^\hmt U = V^\hmt V = I_r$.
       Note that the columns of~$V$ is an orthonormal basis of~$\range(A^\hmt)$.
       Therefore, for any~$x\in \range(A^\hmt)$,
         $\|A x\|= \|UV^\hmt x\| = \|V^\hmt x\| = \|x\|$.
\end{proof}

\begin{theorem}
  \label{th:uniqueevd}
  For any Hermitian matrix~$A \in\CC^{n\times n}$, there exists a unique decomposition
  \begin{equation}
    \label{eq:uniqueevd}
    A \;=\; \sum_{i=1}^k \lambda_i P_i
  \end{equation}
  such that
  \begin{enumerate}
    \item $\{\lambda_i\}$ are all real numbers and~$\lambda_1 > \cdots
      > \lambda_k$.
    \item $\{P_i\}$ are all orthogonal projections, $P_iP_j = 0$ for any distinct~$i$ and~$j$,
      and~$\sum_{i=1}^k P_i = I$.
  \end{enumerate}
\end{theorem}

\begin{proof}
  The existence is easy to establish by any eigenvalue decomposition of~$A$. We only prove the uniqueness.

  Consider any decomposition in the form of~\eqref{eq:uniqueevd}.
  For each~$i$, let~$V_i$ be a matrix whose columns form an orthonormal basis
  of~$\range(P_i)$. Then~$V_i^\hmt V_i$ is an identity matrix, and~$P_i = V_iV_i^\hmt$. For any
  distinct~$i$ and~$j$,
  \begin{equation*}
    V_i^\hmt V_j \;=\;(V_i^\hmt V_i) V_i^\hmt V_j(V_j^\hmt V_j) \;=\; V_i^\hmt P_iP_j V_j\;=\; 0.
  \end{equation*}
  Define
  \begin{equation*}
    V = (V_1\; \cdots \; V_k).
  \end{equation*}
  Then the columns of~$V$ are orthonormal. In addition,
  \begin{equation*}
    VV^\hmt \;=\; \sum_{i=1}^k V_iV_i^\hmt \;=\; \sum_{i=1}^k P_i \;=\; I.
  \end{equation*}
  Thus~$V$ is a unitary matrix. In addition,
  \begin{equation*}
    A \;=\; \sum_{i=1}^k \lambda_i P_i \;=\; \sum_{i=1}^k\lambda_iV_iV_i^\hmt \;=\; \sum_{i=1}^k V_i\Lambda_iV_i^\hmt \;=\; V\Lambda V^\hmt,
  \end{equation*}
  where~$\Lambda_i = \lambda_i V_i^\hmt V_i$, and~$\Lambda$ is the block diagonal matrix whose
  diagonal blocks are~$\Lambda_i$. Note that~$\Lambda$ indeed a diagonal matrix since
  each~$\Lambda_i$ is diagonal. Thus~$V\Lambda V^\hmt$ is an eigenvalue decomposition of~$A$,
  with~$\lambda_1$, $\dots$, $\lambda_k$ being all the distinct eigenvalues, ranked in the descending
  order. Moreover, for
  each~$i$, the columns
  of~$V_i$ form an orthonormal basis of the eigenspace associated with~$\lambda_i$, and
  hence~$P_i$ is the orthogonal projection onto this eigenspace. In this way, $\{\lambda_i\}$
  and~$\{P_i\}$ are uniquely determined by~$A$.
\end{proof}

\begin{theorem}
  \label{th:uniquesvd}
  For any nonzero matrix~$A\in\CC^{m\times n}$, there exists a unique decomposition
  \begin{equation}
    \label{eq:uniquesvd}
    A \;=\; \sum_{i=1}^k \sigma_i A_i
  \end{equation}
  such that
  \begin{enumerate}[leftmargin=1.5em]
    \item $\sigma_1>\cdots >\sigma_k> 0$;
    \item $\{A_i\}$ are all partial isometries, with
    $A_i A_j^\hmt$ and $A_i^\hmt A_j$ both being zero for any distinct $i$ and~$j$.
  \end{enumerate}
\end{theorem}

\begin{proof}
  The existence is easy to establish by any singular value decomposition of~$A$. We only prove the
  uniqueness.

  Consider any decomposition in the form of~\eqref{eq:uniquesvd}.
  Since~$A_i^\hmt A_j = 0$ for any distinct~$i$ and~$j$, we have
  \begin{equation}
    \label{eq:spectdec}
    A^\hmt A \;=\;
    \left(\sum_{i=1}^k \sigma_iA_i\right)^\hmt
    \left(\sum_{i=1}^k \sigma_iA_i\right)
    \;=\; \sum_{i=1}^k \sigma_i^2 A_i^\hmt A_i.
  \end{equation}
  For each~$i$, $A_i^\hmt A_i$ is an orthogonal projection as~$A_i$ is a partial isometry (see~\ref{it:aha} of
  Proposition~\ref{prop:piso}).
  Hence~\eqref{eq:spectdec} is a decomposition specified in Theorem~\ref{th:uniqueevd}. Due to the
  uniqueness part of Theorem~\ref{th:uniqueevd}, $\sigma_1$, $\dots$,
  $\sigma_k$ and~$A_1^\hmt A_1$, $\dots$, $A_k^\hmt A_k$  are uniquely determined by~$A$.

  Now consider any two decompositions in the form of~\eqref{eq:uniquesvd}. According to what is
  proved above, we can formulate the decompositions as
  \begin{equation}
    \label{eq:2dec}
    A \;=\; \sum_{i=1}^k \sigma_i A_i \quad \text{ and } \quad
    A\;=\;\sum_{i=1}^k \sigma_i \tilde{A}_i,
  \end{equation}
  and we have~$A_i^\hmt A_i= \tilde{A}_i^\hmt\tilde{A}_i$ for each~$i$.
  For any distinct~$i$ and~$j$,
  \begin{equation*}
    A_i\tilde{A}_j^\hmt \;=\; (A_iA_i^\hmt A_i)\tilde{A}_j^\hmt
    \;=\; (A_i\tilde{A}_i^\hmt \tilde{A}_i)\tilde{A}_j^\hmt \;=\; 0,
  \end{equation*}
  where the first equality is because~$A_i$ is a partial isometry (see~\ref{it:aaha} of
  Proposition~\ref{prop:piso}), and the second is because~$\tilde{A}_i\tilde{A}_j^\hmt = 0$.
  Similarly, $\tilde{A}_iA_j^\hmt = 0$. Hence
  \begin{equation*}
    (A_i-\tilde{A}_i)(A_j-\tilde{A}_j)^\hmt \;=\; 0.
  \end{equation*}
  Thus
  \begin{equation}
    \label{eq:ddh}
    \left[\sum_{i=1}^k \sigma_i(A_i-\tilde{A}_i) \right]
    \left[\sum_{i=1}^k \sigma_i(A_i-\tilde{A}_i) \right]^\hmt
    \;=\; \sum_{i=1}^k \sigma_i^2 (A_i-\tilde{A}_i)(A_i-\tilde{A}_i)^\hmt.
  \end{equation}
  According to~\eqref{eq:2dec}, the left-hand side of~\eqref{eq:ddh} is zero.
  Hence~$A_i=\tilde{A}_i$ for each~$i$. Therefore, the two decompositions in~\eqref{eq:2dec} are
  identical. The proof is complete.
\end{proof}

Theorem~\ref{th:uniquesvd} is indeed the matrix version of the following theorem.

\begin{theorem}[{\cite[Theorem~1]{Koranyi_2001}}]\label{th:orthodec}
  Suppose that~$X$ and~$Y$ are finite dimensional Hilbert spaces, and~$T\mathrel{:}X\to Y$ is a linear
  operator. Then there exist unique orthogonal decompositions
  \begin{equation*}
    \range(T^*) = X_1 \oplus \cdots \oplus X_k, \quad
    \range(T) = Y_1 \oplus \cdots \oplus Y_k,
  \end{equation*}
  scalars~$\sigma_1>\cdots >\sigma_k$, and isometries~$T_i\mathrel{:}X_i\to Y_i$ \textnormal{(}$i=1,
  \dots, k$\textnormal{)}
  such that
  \begin{equation*}
    T|_{X_i} \;=\; \sigma_i T_i
    \quad \text{ for each }\quad i\in\{1,\dots,k\}.
  \end{equation*}
\end{theorem}


\section{Understanding SVD as a change of basis}

\begin{theorem}
  \label{th:matrix}
  Suppose that~$X$ is finite dimensional vector space on~$\FF$ with $\{x_1,  \dots,
  x_n\}$ being its basis and~$C_\sss{X} : X \to \FF^{n}$ being the map from any point in~$X$ to its
  coordinate under this basis; $Y$, $\{y_1, \dots, y_m\}$, and~$C_\sss{Y} : Y \to \FF^{m}$ are similar.
  Consider a linear operator~$T \mathrel{:} X \to Y$.
  \begin{enumerate}[leftmargin=1.5em]
    \item There is a unique matrix~$A\in \FF^{m\times n}$
   that represents~$T$ under the aforementioned bases of~$X$ and~$Y$ in the sense that
  \begin{equation*}
    A C_\sss{X} (x) = C_\sss{Y} (T (x)) \quad \text{ for all } \quad x\in X,
  \end{equation*}
  meaning that applying~$T$ to~$x$ is equivalent to multiplying its coordinate by~$A$.
  Indeed, the $i$-th column of~$A$ is $A_i = C_\sss{Y}(T(x_i))$, namely the coordinate of~$T(x_i)$.
%\item $T = C_\sss{Y}^\inv A C_\sss{X}$, meaning that applying~$T$ to any vector in~$X$ is equivalent to
%  multiplying its coordinate by~$A$ and then using the result as the coordinate to locate a vector in~$Y$.
%\item $C_\sss{X}(\ker(T)) = \ker(A)$, and~$C_\sss{Y} (\range(T)) = \range(A)$.
\item $A$ has full row rank if and only if~$T$ is injective; $A$ has full column rank if and only
  if~$T$ is surjective;
  when~$m=n$, $T$ is invertible if and only if~$A$ is invertible, and~$A^\inv$
  represents~$T^\inv\mathrel{:} Y\to X$ under the aforementioned bases for~$X$ and~$Y$.
\item $A^\adj$ represents~$T^\adj \mathrel{:} Y^\adj \to X^\adj$ under the bases for~$X^*$ and~$Y^*$
  that are dual to~$\{x_1, \dots, x_n\}$ and~$\{y_1, \dots, y_n\}$ respectively.
\item Let~$\{x_1', \dots, x_n'\}$ be a basis for~$X$, $\{y_1', \dots, y_m'\}$ be a
  basis for~$Y$, and~$B\in \FF^{m\times n}$ be the representation of~$T$ under such bases.
  Then~$A = Q^\inv B P$, where~$P\in \FF^{n\times n}$ and~$Q\in \FF^{m \times m}$
  are the changing of basis matrices such that
  \begin{equation*}
    (x_1\; \cdots\; x_n) = (x_1'\; \cdots\; x_n')P, \qquad
    (y_1\; \cdots\; y_m) = (y_1'\; \cdots\; y_m')Q.
  \end{equation*}
  \end{enumerate}
\end{theorem}

With the view point presented in~Theorem~\ref{th:matrix}, we can understand SVD as follows.

Let~$A\in \CC^{m\times n}$ be a nonzero matrix, and~$U\Sigma V^\hmt$ be its SVD.
Consider the linear operator~$T \mathrel{:} x \mapsto A x$ from~$\CC^{n}$ to~$\CC^{m}$.
Then~$A$ represents~$T$ under the canonical bases. If we take the columns of~$V$ as the basis
for~$\CC^{n}$ and those of~$U$ as the basis for~$\CC^{m}$, then SVD provides a simple
representation for~$T$, which is~$\Sigma$.

%Suppose that~$\hat{U}\hat{\Sigma} \hat{V}^\hmt$ is a compact SVD of~$A$.
%Then~$A \hat{V} = \hat{U}\hat{\Sigma}$.
%Note that the columns of~$\hat{U}\hat{\Sigma}$ form an orthogonal (not necessarily normalized) basis
%for~$\range(A)$, and those of~$\hat{V}$ form an orthonormal basis for~$\range(A^\hmt)$.  Thus
%SVD tells us that there exists an orthogonal basis for~$\range(A^\hmt)$ such that its image
%under~$T\mathrel{:}x\mapsto Ax$ is an orthogonal basis for~$\range(A)$.

Recall the decompositions
\begin{equation*}
\CC^{n} \;=\; \ker(A) \oplus \range(A^\hmt),
\quad
\CC^{m} \;=\; \ker(A^\hmt) \oplus \range(A).
\end{equation*}
When~$T$ acts on~$\CC^{n}$, it drops out the information in~$\ker(A)$, and provides no information
in~$\ker(A^\hmt)$.  Consequently,~$T$ is not an isomorphism if either~$\ker(A)$ or~$\ker(A^\hmt)$
is nonzero.
The restriction~$\hat{T} \mathrel{:} \range(A^\hmt)\to\range(A)$ with~$\hat{T} (x) =Ax$ is however
always an isomorphism.
Suppose that~$\hat{U}\hat{\Sigma} \hat{V}^\hmt$ is a compact SVD of~$A$.
Note that the columns of~$\hat{U}$ form an orthogonal basis
for~$\range(A)$, and those of~$\hat{V}$ form an orthonormal basis for~$\range(A^\hmt)$.
Under these bases,~$\hat{T}$ is represented by~$\hat{\Sigma}$.
The representation for~$\hat{T}^\inv\mathrel{:} \range(A)\to \range(A^\hmt)$ be
is~$\hat{\Sigma}^\inv$.

The operator~$T^\pin \mathrel{:} \CC^{m}\to \CC^{n}$~defined by%
\[
T^\pin|_{\range(A)} = \hat{T}^\inv, \quad
T^\pin|_{\ker(A^\hmt)} = 0
\]
is called the Moore-Penrose pseudoinverse of~$T$.
The representation
of~$T^\pin$ under the canonical bases is called the Moore-Penrose pseudoinverse of~$A$, which turns
out to be~$A^\pin = \hat{U}\hat{\Sigma}^\inv\hat{V}^\hmt$.

We can regard any nonzero linear operator as a bijection by restricting its domain and image
space.
SVD tells us that any nonzero linear operator between finite dimensional Hilbert spaces
can be represented by a positive diagonal matrix under properly chosen orthonormal bases for the
restricted domain and image space. This can also lead us to Theorem~\ref{th:orthodec}.


\section{Examples of applications}

\begin{proposition}[Polar decomposition]
  \label{th:polar} Let~$A \in \CC^{m\times n}$ be a matrix.
  \begin{enumerate}[leftmargin=1.5em]
    \item If~$m\ge n$, there exists a positive semidefinite matrix~$P \in \CC^{n\times n}$ and
      a matrix~$U\in \CC^{m\times n}$ such that~$A = UP$ and~$U^\hmt U = I_n$; there also exists
      a positive semidefinite matrix~$Q\in \CC^{m\times m}$ and a matrix~$V\in\CC^{m\times n}$ such
      that~$A = QV$ and~$V^\hmt V = I_n$.
      In this case, $P = (A^\hmt A)^{\frac{1}{2}}$.
    \item If~$n\ge m$, there exists a positive semidefinite matrix~$P \in \CC^{n\times n}$ and
      a matrix~$U\in \CC^{m\times n}$ such that~$A = UP$ and~$UU^\hmt = I_m$; there also exists
      a positive semidefinite matrix~$Q\in \CC^{m\times m}$ and a matrix~$V\in\CC^{m\times n}$ such
      that~$A = QV$ and~$V V^\hmt = I_m$. In this case, $Q = (AA^\hmt)^{\frac{1}{2}}$.
  \end{enumerate}
   If~$A$ is real, we can require~$P$, $U$, $Q$, and~$V$ to be real.
\end{proposition}

\begin{proof}
  We only prove~1.
  Let~$ W\Sigma Z^\hmt$ be an SVD of~$A$.

  Note that the last~$m-n$ rows of~$\Sigma$ are zero.
  Let~$\hat{\Sigma}$ be the first~$n$ rows of~$\Sigma$, and~$\hat{W}$ be the first~$n$ columns
  of~$W$. Then~$A = \hat{W}\hat{\Sigma}Z^\hmt $. Define~$U = \hat{W}Z^\hmt$ and~$P
  = Z\hat{\Sigma} Z^\hmt$. Then~$P$ is positive semidefinite, and
  \begin{equation*}
    A \;=\; \hat{W}\hat{\Sigma} Z\;=\; UP,\quad
    U^\hmt U \;=\; Z\hat{W}^\hmt \hat{W} Z^\hmt \;=\;ZZ^\hmt \;=\; I_n.
  \end{equation*}
  Consequently, $A^\hmt A  =  P^\hmt U^\hmt U P = P^2$, and hence~$P = (A^\hmt
  A)^{\frac{1}{2}}$.

  Let~$\bar{\Sigma} = (\Sigma\;\, 0_{m\times(m-n)})$ and~$\bar{Z} = (Z\;\,0_{n\times (m-n)})$.
  Then~$A = W\bar{\Sigma} \bar{Z}^\hmt$. Define~$Q = W\bar{\Sigma} W^\hmt$ and~$V = W\bar{Z}^\hmt$.
  Then~$Q$ is positive semidefinite, and
  \begin{equation*}
    A\;=\; W\bar{\Sigma}\bar{Z}^\hmt \;=\; QV, \quad
    V^\hmt V \;=\; \bar{Z}W^\hmt W \bar{Z}^\hmt \;=\; \bar{Z}\bar{Z}^\hmt \;=\; ZZ^\hmt \;=\; I_n.
  \end{equation*}

  If~$A$ is real, then~$W$, $\Sigma$, and~$Z$ can all be real,
  ensuring~$P$, $U$, $Q$, and~$V$ to be real.
\end{proof}

\begin{proposition}[\cite{Fan_Hoffman_1955}]
  Let~$H$ be the Hermitian part of a matrix~$A \in \CC^{n\times n}$. Enumerating the eigenvalues
  of~$H$ as~$\lambda_1(H) \ge\dots\ge \lambda_n(H)$, and the singular values of~$A$
  as~$\sigma_i(A) \ge\dots\ge \sigma_n(A)$, we have~$\sigma_{i}(A)\ge
  \lambda_i(H)$ for each~$i = 1, \dots, n$.
\end{proposition}

\begin{proof}
  By Theorem~\ref{th:polar}, there exists a positive semidefinite matrix~$P\in \CC^{n\times n}$ and
  a unitary matrix~$U\in \CC^{n\times n}$ such that~$A = UP$. For any unit vector~$x\in \CC^n$,
  \begin{equation*}
    x^\hmt H x = \frac{1}{2}x^\hmt (A^\hmt + A)x = \Re(x^\hmt A x) \le %|x^\hmt A x| =
    |x^\hmt U P x|\le \|Px\| = (x^\hmt P^2 x)^{\frac{1}{2}} =(x^\hmt A^\hmt A x)^{\frac{1}{2}}.
  \end{equation*}
  Therefore, by the Courant-Fischer-Weyl min-max principle, we know that
  \begin{equation*}
    \lambda_i(H) \;\le\; \lambda_i(A^\hmt A)^{\frac{1}{2}}\;=\; \sigma_i(A).
  \end{equation*}
\end{proof}

\begin{proposition}
  For matrices~$A_1$ and $A_2 \in \CC^{m\times n}$, $A_1^\hmt A_1 = A_2^\hmt A_2$ if and only if there exists a unitary
  matrix~$U \in \CC^{m\times m}$ such that~$A_2 = UA_1$.
\end{proposition}

\begin{proof}
  The ``if'' part is trivial. We focus on the ``only if'' part.
  Let~$V \Lambda V^\hmt$ be an eigenvalue decomposition of~$A_1^\hmt A_1 = A_2^\hmt A_2$ such that
  the diagonal entries of~$\Lambda$ are descending. By
  Corollary~\ref{coro:eyfull}, there exists $W_1$, $W_2\in \CC^{m\times m}$ and~$\Sigma \in \RR^{m\times n}$
  such that~$A_1=W_1 \Sigma V^\hmt$
  and~$A_2 = W_2 \Sigma V^\hmt$. Set~$U = W_2W_1^\hmt$.
\end{proof}

\small
\bibliography{\bibfile}
\bibliographystyle{plain}
\end{document}
